\documentclass{beamer}
\usepackage{graphicx}
\usepackage{wrapfig}

\graphicspath{ {images/} }
\usetheme{Madrid}
\usecolortheme{wolverine}

\title[Project A]
{Axion Searches with Microwave-Cavity}
\author[Hunc]{Hu Naichao}
\institute[SPE]{Physics Department\\
School of Physics and Engineering\\Sun Yat-Sen University}
\date{May 2014}

\logo{\includegraphics[height=1.5cm]{ulogo}}

\begin{document}

\frame{\titlepage}

%genarates table of contents
\begin{frame}
\frametitle{Table of Contents}
\tableofcontents
\end{frame}
%genarates content of current section
\AtBeginSection[]
{
  \begin{frame}
    \frametitle{Table of Contents}
    \tableofcontents[currentsection]
  \end{frame}
}

%main part
%section I========================================================
\section{Overview}
\begin{frame}
\frametitle{Brief Introduction}

\begin{wrapfigure}{r}{0.25\textwidth}
    \centering
    \includegraphics[width=0.25\textwidth]{NPieCh}
    \caption{Breakdown of our universe}
\end{wrapfigure}

The advent of precision cosmology provides better understanding of where we stands.
\begin{itemize}
\item <1->The Universe is flat to high degree and\\ $\Omega_{dm} \simeq 0.3$
\item <2->CDM candidates emerge as necessary \\component to explain LSS, among which\\ axions is a "usual suspects"
\item <3->More than 80\% of matter is DM 
\end{itemize}
\end{frame}

\begin{frame}{Experimental Evolution}
\begin{block}{Note}
We assume the axions exists and dominate our galactic halo. This  presentation is dedicated to \alert{macrowave cavity} searches for axions.
\end{block}
An elegant technique has been developed by Sikivie, used and modified by many experiments.
\begin{itemize}
\item<1->Conventional amplifier
Based on heterostructure field-effect transistors(HFET's), aka, high-electron-mobility transistors(HEMT).\\
Noise $\sim 1.5 K$ at $GHz$ range frequency.
    \begin{itemize}
    \item Sikivie macrowave cavity experiment
    \item The U.S. large-scale search (ADMX 1995)
    \end{itemize}
\item<2-> High-gain gigaherz dc Superconducting Quantum Interference Device (SQUID)\\
Noise $\sim 50 mK$ at $600-MHz$ frequency (about twice SQL).
       %a possible 3 generation search ex
\item<3->Single photon detector
Rydberg-atom single-quantum detector.\\
Noise as low as $\sim 12 mK$, eviding SQL entirely.
    \begin{itemize}
    \item The CARRACK1 detector
    \end{itemize}
\end{itemize}
\end{frame}

%section II=======================================================
\section{The Axion Theory}
%page 1
\begin{frame}{From Strong-CP Problem}
Postulated roughly 4 decades ago, axion elegantly solves the "strong-CP" problem by presenting a quasisymmetry $U_{PQ}(1)$, i.e.
$$\bar{\theta} = \theta - arg \, det(m_1,m_2, \ldots, m_n) - \frac{a(x)}{f_a},$$
where $a(x)$ is the axion field, $f_a = v_a/N$ is called the axion decay constant, $v_a$ is the vacuum expectation value,
which spontaneously breaks $U_{PQ}(1)$, and $N$ is an integer that expresses the color anomaly of $U_{PQ}(1)$ symmetry.
\\
\vspace{1cm}
The nonperturbative effects then produce a effective potential $V(\bar{\theta})$, whose minimum is at $\bar{\theta} = 0$.
\end{frame}
%page 2
\begin{frame}{Axion Parameters}
The axion mass is given in terms of $f_a$ by$$
m_a \simeq 0.6 eV \frac{10^7GeV}{f_a}.$$
All the axion couplings are inversely proportional to $f_a$.
\\One in particular is the axion coupling to two
photons:$$
\mathcal{L}_{a \gamma \gamma} = -(\frac{\alpha}{\pi} \frac{g_{\gamma}}{f_a}) a \vec{E} \vec{B},$$
where $\vec{E}$ and $\vec{B}$ are the usual electric and magnetic
fields, $\alpha$ is the fine-structure constant, and $g_{\gamma}$
is a modeldependent coefficient of order 1.
\end{frame}
%page 3 
\begin{frame}{A Brief on Constraints}
A priori, the value of $v_a$, and thus $m_a$, is arbitrary. At present, the combined limits rule out axions heavier than about $3 \times 10^{-3}eV$.
\begin{itemize}
\item If the axion is heavier than $1 MeV$, and decays quickly into $e^+ e^-$ (lifetime of order $10^{-11} sec$ or less), then it is ruled out by negative searches for rare particle decays such as $\pi^+ \rightarrow a(e^+ e^-)+e^++\nu_e$, whose rates are well known.
\item If lifetime is greater than $10^{-11} sec$, they are severely constrained by negative searches in beam dumps. Since in a beam dump, many different process involving independent couplings should add up incoherently and can't all vanish. Many such experiment thus ruled out mass range down to about $50 keV$. 
\end{itemize}
\end{frame}
%page 4
\begin{frame}{A Brief on Constraints}
\begin{itemize}
\item Emission of no-rescattering axions provides an efficient way of cooling stars. Thus the abundance of red giants ruled out the mass range $0.5 eV \lesssim m_a \lesssim 200keV$ for KSVZ type. This range extends as the coupling to $e^-$ grows.
\item The terrestrial neutrino signal from SN 1987a eliminates the range $3 \times 10^{-3} eV \lesssim m_a \lesssim 2 eV$. The observation is consistent with theoretical expectations based on the premise that the core cools solely by emission of neutrinos suggests that axions need to be very weekly coupled. (NB: This constraint is relatively axion model independent.)
\end{itemize}
\end{frame}
%page 5
\begin{frame}{Axions and Cosmology}
At a critical time $t_1 = m_a^{-1}(t_1)$, temperature of order $f_a (\sim 1 GeV)$ , a phase transition occurs in which the $U_{PQ}(1)$ symmetry becomes spontaneously broken. This is called the PQ phase transition, when the nonperturbative effects that produce the effective potential $V(\bar{\theta})$ are suppressed, the axion is massless,
and all values of $\langle a(x) \rangle$ are equally likely. After the transition, the axion field starts to oscillate in response to the axion mass turn-on.\\
We distinguish two cases:
\begin{enumerate}
\item Inflation occurs with reheat temperature less than the PQ transition temperature;
\item inflation occurs with reheat temperature higher than the PQ transition temperature (equivalently, for our purposes, inflation does not occur at all). 
\end{enumerate}
\end{frame}
%page 6
\begin{frame}{Case I}
Homogenized by inflation, the initial amplitude depends on how far from zero the axion field is at $t_1$. The contribution to $\Omega_0$ in terms of $\rho_c$ is$$
\Omega_a = \frac{\rho_a(t_0)}{\rho_c} = \frac{1}{6}\alpha^2(t_1)(\frac{f_a}{10^{12}GeV})^{7/6}(\frac{0.7}{h_{100}})^2,$$
where $\alpha(t_1) = a(t_1)/f_a$ is the initial misaligment angle.\\
\vspace{1cm}
NB: $\Omega_a$ can be supressed if $a(t_1) \sim 0$. \\
NNB: This contribution is also called \alert{vacuum realignment}.
\end{frame}
%page 7
\begin{frame}{Case II}
An reasonable assumption, in this case, would be after $t_1$, each axion string turns into boundary of \alert{1} domain wall. Thus  contributions to $\Omega_0$ are as follows: (1)from axions that were radiated before $t_1$; (2)from the decay of walls after $t_1$; (3)from vacuum realignment.\\
Now if we introduce a parameter $\xi$ to discribe the average distance between axion strings at time $t$ as $\xi t$, and if we take $\xi = 1$, we have\begin{equation*}
\begin{split}
\Omega_a &= \Omega^{vac}+\Omega^{d.w.}+\Omega^{str}\\
&= (0.5-3.0)(\frac{f_a}{10^{12}})^{7/6}(\frac{0.7}{h_{100}})^2
\end{split}
\end{equation*}
\end{frame}
%page 8
\begin{frame}{Summary of Constraints}
If we plug in known $\Omega_0$ as $\Omega_a$, an open mass range at $1-100 \mu eV$ is obtained.\\
Combining privious constraints, we have the following fig:
\begin{figure}
\centering
\includegraphics{range}
\caption{Ranges of axion mass $m_a$ (or $f_a$)}
\end{figure}
\end{frame}
%page 9
%\begin{frame}{Phase-space structure of halo dark-matter axions}
%not understood 
%\end{frame}

%section III======================================================
\section{The Sikivie Experiment}
%page 1
\begin{frame}{Primakoff Effect}
The Primakoff effect is the resonant production of neutral pseudoscalar mesons by high-energy photons interacting with an atomic nucleus. \\
For axions, in a strong magnetic field, their convertion to photons can be detected theoretically.  
\begin{figure}
    \caption{Feynman diagram for Primakoff effect, where $e$ for electrons, $I$ for ions.}
    \centering
    \includegraphics[width=0.5\textwidth]{prima}
\end{figure}
\end{frame}
%page 2
\begin{frame}{General Detector Properties}
When a frequency $\omega=2 \pi f$ of a cavity mode equals $m_a$, axions convert resonantly into photons. The power from axion-to-photon conversion
on resonance is found to be
\begin{equation*}
\begin{split}
P_a &= (\frac{\alpha}{\pi} \frac{g_\gamma}{f_a})^2 V B_0^2 \rho_a C \frac{1}{m_a} Min(Q_L,Q_a)\\
&=0.5\times10^{-26}W(\frac{V}{500 liter})(\frac{B_0}{7 T})^2 C (\frac{g_\gamma}{0.36})^2\\
&\times (\frac{\rho_a}{\frac{1}{2}\times10^{-24}g/cm^3})\times (\frac{m_a}{2\pi(GHz)})Min(Q_L,Q_a),
\end{split}
\end{equation*}
where $V$ is the volume if the cavity, $Q_L$ is the cavity's loaded quality factor, $Q_a = 10^{16}$ is the Q factor for galactic halo axion signal, $\rho_a$ is the density of galactic halo axions on Earth, $C$ is a mode-dependent form factor.\\
For a cylindrical cavity with  the $TM_{010}$ mode yields the largest form factor with $C \approx 0.69$.
\end{frame}
%page 3
\begin{frame}{General Detector Properties}
As a cavity scans over the
possible axion mass range, the scan rate (aka, SNR) is determined
by the amount of time it takes for a possible axion signal
to be detected over the microwave cavity’s intrinsic noise, or mathematically,$$
SNR = \frac{P_a}{\bar{P_N}}\sqrt{Bt} = \frac{P_a}{k_BT_S}\sqrt{\frac{t}{B}},$$
where $\bar{P_N}= k_BBT_S$ is the cavity noise power, $B$ is the bandwidth, and $T_S$
is the sum of the physical
temperature of the cavity plus the noise temperature of
the microwave receiver.\\
At a given SNR, we have a useful logarithmic scan rate relation as follows,
$$\frac{1}{f}\frac{df}{dt} \propto (B_0^2V)^2\frac{1}{T_S^2}.$$
\end{frame}
%page 4
\begin{frame}{First-gerneration Experiments}
First-g eneration microwave cavity experiments demonstrated the
feasibility of the technique over a significant range of
frequencies, but fell short in power sensitivity by 100–
1000 in efforts to detect halo axions with plausible model couplings, due to their \alert{small volumes}
and relatively \alert{high noise temperatures}.\\
Brief of two experiment:
\begin{enumerate}
\item The Rochester-Brookhaven-Fermilab experiment\\
\begin{itemize}
\item peak central field of $8.5T$; $Q_W \sim 1.8 \times 10^5 \,(unloaded\,at\,2-3GHz,\,4.2K)$; higher-TM mode; continuously swept.
\end{itemize}
\item The University of Florida experiment
\begin{itemize}
\item peak central field of $8.6T$; $Q_W \sim 1.6 \times 10^5 \,(unloaded\,at\,1.5GHz,\,4.2K)$; tuned in steps
\end{itemize}
\end{enumerate}
\end{frame}
%page 5
\begin{frame}{First-gerneration Experiments}
\begin{itemize}
\item RBF Expriment: Limits ($95\%$ C.L.) on $g_{a\gamma\gamma}^2 GeV^2$
were established of $5.7 \times 10^{-28}$ at the lowmass end ($4.5\mu eV$) to $1.8\times10^{-25}$ on the high-mass end ($16\mu eV$).
\item UF Expriment: Limits ($97.5\%$ C.L.) on $g_{a\gamma\gamma}^2 GeV^2$
were established of $3.3 \times 10^{-28}$ at the lowmass end ($5.46-5.95\mu eV$) to $1.6\times10^{-27}$ on the high-mass end 
($7.46-7.60\mu eV$).
\end{itemize}
\begin{figure} 
    \centering
    \includegraphics[height=3.4cm]{fger}
    \caption{Exclusion regions for both expriments}
\end{figure}
\end{frame}

%section IV=======================================================
\section{The U.S. Large Scale Search}
%page 1
\begin{frame}{ADMX(1995)-Hardware}

\end{frame}

%section V========================================================
\section{SQUID Amplifier \& Third Genaration Searching}

%section VI=======================================================
\section{Rydberg-Atom \& The CARRACK Experiment}
% page 1
\begin{frame}
  \frametitle{The CARRACK Scheme}
The CARRACK (Cosmic Axion Research using Rydberg Atoms in a resonant Cavity in
Kyoto) experiment, rather than amplifiers, opted to develop a Rydberg-atom
based single-quantum detector.
  \begin{figure}
    \centering
    \includegraphics[width=10cm]{jsche}
    \caption{General schematic of CARRACK experiment utilizing Rydberg atoms
      to recover single photons generated in the microwave cavity.}
  \end{figure}
\end{frame}
% page 2
\begin{frame}
  \frametitle{Rydberg Atoms}
  Currently studied Rydberg atoms experimentally have $10<n<150$, which makes
transitions between states fall in the microwave region ($\Delta E_{100} \sim 7GHz$).
  \begin{itemize}
  \item The energy levels of nonhydrogenic alkaki atoms, which depend on the angular momentum $\ell$
    through the number $\delta_{\ell}$ are given
    by $$E_{n,\ell}=\frac{\mathcal{R}}{(n-\delta_{\ell})^2}=\frac{\mathcal{R}}{\tilde{n}^2}.$$
  \item As a result of the coupling of all of the atoms to the spatially
    coherent field, a collective atom systom formed. Due to
    which, the coupling $\Omega$ between the cavity and the atoms is replaced
    by $\Omega_N=\Omega \sqrt{N}$.
  \end{itemize}
\end{frame}
% page 3
\begin{frame}
  \frametitle{Field Ionization}
not understood yet!  
\end{frame}
% page 4,5,6,7
\begin{frame}
  \frametitle{The CARRACK1 Detector}
  The system has been utilized to search for axions at $10\mu eV$ mass region.
  \begin{itemize}
  \item<1-> Apparatus
    \begin{onlyenv}<1>
      \begin{itemize}
      \item Two coupled cavity, one for convertion, the other detection
      \item Cooled to roughly $15mk$
      \end{itemize}
    \end{onlyenv} 
  \item<2-> Frequency tuning
    \begin{onlyenv}<2>
      \begin{itemize}
      \item Accomplished by inserting dielectric rods along the axis 
      \end{itemize}
      \begin{figure}
        \centering
        \includegraphics[width=4cm]{crodf}
        \caption{Effects of tuning rod in the convertion cavity}
      \end{figure}
     \end{onlyenv}
   \item<3-> Magnetic-field shielding in the detection cavity
     \begin{onlyenv}<3>
       \begin{itemize}
       \item Cancellation of a superconducting coil
       \item Demagnetization via Meissner effect
       \end{itemize}
     \end{onlyenv}
   \item<4-> Selective field-ionization detector
     \begin{onlyenv}<4>
       \begin{itemize}
       \item the field-ionization electrodes, transport electrodes for
         electrons and an electron multiplier
       \item To make the mutiplier work properly, it needs to be heated to at
         least $20K$. (Isolated by TI polymer.)
       \end{itemize}
     \end{onlyenv}
  \end{itemize}
\end{frame}
% page 8,9
\begin{frame}
  \frametitle{The CARRACK2 Detector}
  \begin{onlyenv}<1>
  \begin{wrapfigure}{l}{0.33\textwidth}
    \centering
    %\caption{Schematic drawing of CARRACK2}
    \includegraphics[width=0.33\textwidth]{carr2}
  \end{wrapfigure}
  Following experience obtained and more thorough theoretical and numerical
  analysis, CARRACK2 was designed and constructed.
  \begin{itemize}
  \item The region to be covered is from $\sim 2\mu eV$ to $\sim 30\mu eV$.
    \begin{itemize}
    \item For masses less than $2\mu eV$, $n \geq 180$, field-ionization
      method find it impossible to separate excited states.
    \end{itemize}
  \item Maximum field $7T$.
  \end{itemize}
  \end{onlyenv}
  \begin{onlyenv}<2>
    Major modifications are listed as follows:
    \begin{enumerate}
    \item Instead of $Rb$, Potassium Rydberg atoms will be used (reduce the
      effect of Stark broadening in the microwave absorption process).
    \item Guiding field method to avoid the rotation of electric field and
      increase absorption probability of photons.
    \item A spatially collimated bunched packets of Rydberg atomic beam (by
      laser cooling) time varying electric field will be applied to compensate
      the stray electric field.
    \end{enumerate}
  \end{onlyenv}
\end{frame}

%section VII======================================================
\section{A Summary}
%page 1
\begin{frame}
  \frametitle{Summary}
  Although the expected axion convertion power is remarkably weak, the
  technology development so far has been very promising.
  \begin{itemize}
  \item The new microstrip technique for coupling the signal into a SQUID has
    extended the application of SQUID’s as high-gain amplifiers into the
    $1-GHz$ region. And new SQUID layouts for higher frequency are under
    consideration. 
  \item Higher-frequency operation poses no new conceptual problem for the
    Rydberg-atom single-quantum technique. 
  \end{itemize}
  Taking a optimistic point of view,  the technology will soon be in hand to
  either find the axion with high probability, if it exists, or if not,
  exclude it.
\end{frame}
%page 2
\begin{frame}
  \frametitle{Outlook}
  
\end{frame}
\end{document}
