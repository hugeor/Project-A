\documentclass{beamer}
\usepackage[utf8]{inputenc}

\usepackage{utopia} %font utopia imported
\usefonttheme{professionalfonts}
\usepackage{graphicx}
\usepackage{wrapfig}

\graphicspath{ {images/} }

\usetheme{Antibes}
\usecolortheme{beaver}

\usepackage{xeCJK}
\setCJKmainfont{simsun.ttf}
\setCJKsansfont{simhei.ttf}
\setCJKmonofont{simfang.ttf}

%------------------------------------------------------------
%This block of code defines the information to appear in the
%Title page
\title[About] %optional
{The Road to Reality}

\subtitle{A Brief Guide to Active Endevour of Last Year}

\author[Hunc]{胡乃超}
\institute[SPE]{Department of Physics\\
School of Physics and Engineering}


\date[SYSU 2014] % (optional)
{引力波楼, Sep 2014}

%\logo{\includegraphics[height=1.5cm]{lion-logo.png}}

%End of title page configuration block
%------------------------------------------------------------



%------------------------------------------------------------
%The next block of commands puts the table of contents at the 
%beginning of each section and highlights the current section:

\AtBeginSection[]
{
  \begin{frame}
    \frametitle{Table of Contents}
    \tableofcontents[currentsection]
  \end{frame}
}
%------------------------------------------------------------


\begin{document}

%The next statement creates the title page.
\frame{\titlepage}


%---------------------------------------------------------
%This block of code is for the table of contents after
%the title page
\begin{frame}
\frametitle{Table of Contents}
\tableofcontents
\end{frame}
%---------------------------------------------------------

%=========================================================
\section{Individualities and personal tastes}

%---------------------------------------------------------
\begin{frame}
  \frametitle{A prelude}
  I'd like to make some statements first to explain why I find
  the fundamental part of advanced physics more intriguing than
  others.\par 

  \begin{itemize}
  \item<1-> Ignoring the lack of understanding to foundations of
    physics is unsettling.% That could keep one awake at night.
  \item<2-> Theories, like general relativity, that carries no less
    beauty than ``Euclid's Elements'' is more than tempting.
  \item<3-> Since I have decided to become nerdy, what's to prevent me
    from going further?
  \end{itemize}
  
\end{frame}

%---------------------------------------------------------
%=========================================================


%=========================================================
\section{Several expeditions undertaken in physics}

\subsection{Theoretical physics}
%---------------------------------------------------------
\begin{frame}
  \frametitle{Cosmology}
    \begin{alertblock}{Friedmann equation}
      \begin{onlyenv}<1>
        $$H^2 = \frac{8\pi G}{3}\epsilon-\frac{k}{R_0^2a^2}$$
      \end{onlyenv}
      \begin{onlyenv}<2-3>
        $$\frac{H^2}{H_0^2} =
        \frac{\Omega_{r,0}}{a^4} + \frac{\Omega_{m,0}}{a^3} +
        \Omega_{\Lambda,0} + \frac{1-\Omega_0}{a^2}$$
      \end{onlyenv}
  \end{alertblock}
  
  \begin{itemize}
  \item<3> Large scale uniformity (horizon problem)
  \item<3> Flatness problem
  \item<3> The monopole problem
  \end{itemize}
  The \textit{inflation theory} solves the above problems with a natural
  approach.

\end{frame}

%\begin{frame}
 % \frametitle{Particle physics}
  %Something about spinors.
  %Book : Griffins Introduction to particle physics.\par  
  %Maybe I should not mention any of it here.
  %
%\end{frame}

\begin{frame}
  \frametitle{String theory: compactification}
  \begin{alertblock}{Poission equation in D-dimensional spacetime}
    $$\nabla^2V_g^{(D)} = 4\pi G^{(D)}\rho_m$$ 
  \end{alertblock}
  \begin{columns}
    \column{0.5\textwidth}
    By maintaining the dimension of $G^{(D)}\rho_m$, one obtain
    $$(\ell_P^{(D)})^{D-2} = (\ell_P)^2\frac{G^{(D)}}{G}.$$

    \column{0.5\textwidth}
    Place a mass distribution in extra dimensions, and then 
    $$\frac{G^{(D)}}{G} = V_C.$$
  \end{columns} \pause
  Combining the groundwork and solving for $\ell_C$, we have
  $$\ell_C = \ell_P^{(D)}(\frac{\ell_P^{(D)}}{\ell_P})^{\frac{2}{D-4}}.$$
  
\end{frame}
%---------------------------------------------------------

\subsection{Computational physics}
%---------------------------------------------------------
\begin{frame}
  \frametitle{Algorithms and computations}
  \begin{columns}
    \column{0.5\textwidth}
    This is a text in first column.
    $$E=mc^2$$
    \begin{itemize}
    \item First item
    \item Second item
    \end{itemize}

    \column{0.5\textwidth}
    This text will be in the second column
    and on a second tought this is a nice looking
    layout in some cases.
  \end{columns}
  
\end{frame}
%---------------------------------------------------------

\subsection{Experimental physics}
%---------------------------------------------------------
\begin{frame}
  \frametitle{Axions}
  \begin{itemize}
  \item<1-> Sikivie Experiment (1983)
    \begin{onlyenv}<1>
      \begin{figure}
        \caption{Feynman diagram for Primakoff effect, where $e$ for
          electrons, $I$ for ions.} 
        \centering
        \includegraphics[width=0.5\textwidth]{prima}
      \end{figure}
    \end{onlyenv}
  \item<2-> CARRACK Design (2001)
    \begin{onlyenv}<2>
      \begin{figure}
        \centering
    \includegraphics[width=0.8\textwidth]{jsche}
    \caption{Schematic of CARRACK experiment utilizing Rydberg
      atoms to recover photons generated in the microwave
      cavity.} 
      \end{figure}
    \end{onlyenv}
  \item<3-> New ideas? More stable detectors?
  \end{itemize}
\end{frame}
%---------------------------------------------------------

%=========================================================
%Thanks
\section{}
\begin{frame}
  \begin{center}
    \Huge{\alert{Thanks for your attention!}}\par
    \Huge{\textit{Any questions?}}
  \end{center}
  
\end{frame}

\end{document}
