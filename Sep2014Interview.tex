\documentclass{beamer}
\usepackage[utf8]{inputenc}

\usepackage{utopia} %font utopia imported

\usepackage{graphicx}
\usepackage{wrapfig}

\graphicspath{ {images/} }

\usetheme{Antibes}
\usecolortheme{beaver}

\usepackage{xeCJK}
\setCJKmainfont{simsun.ttf}
\setCJKsansfont{simhei.ttf}
\setCJKmonofont{simfang.ttf}

%------------------------------------------------------------
%This block of code defines the information to appear in the
%Title page
\title[About] %optional
{The Road to Reality}

\subtitle{A Brief Guide to Active Endevour of Last Year}

\author[Hunc]{胡乃超}
\institute[SPE]{Department of Physics\\
School of Physics and Engineering}


\date[SYSU 2014] % (optional)
{引力波楼, Sep 2014}

%\logo{\includegraphics[height=1.5cm]{lion-logo.png}}

%End of title page configuration block
%------------------------------------------------------------



%------------------------------------------------------------
%The next block of commands puts the table of contents at the 
%beginning of each section and highlights the current section:

\AtBeginSection[]
{
  \begin{frame}
    \frametitle{Table of Contents}
    \tableofcontents[currentsection]
  \end{frame}
}
%------------------------------------------------------------


\begin{document}

%The next statement creates the title page.
\frame{\titlepage}


%---------------------------------------------------------
%This block of code is for the table of contents after
%the title page
\begin{frame}
\frametitle{Table of Contents}
\tableofcontents
\end{frame}
%---------------------------------------------------------


\section{Individualities and personal tastes}

%---------------------------------------------------------
%Changing visivility of the text
\begin{frame}
\frametitle{Sample frame title}
This is a text in second frame. For the sake of showing an example.

\begin{itemize}
    \item<1-> Text visible on slide 1
    \item<2-> Text visible on slide 2
    \item<3> Text visible on slides 3
    \item<4-> Text visible on slide 4
\end{itemize}
\end{frame}

%---------------------------------------------------------


%---------------------------------------------------------
%Example of the \pause command
\begin{frame}
In this slide \pause

the text will be partially visible \pause

And finally everything will be there
\end{frame}
%---------------------------------------------------------

\section{Several expeditions undertaken in physics}
\subsection{Theoretical physics}
%---------------------------------------------------------
%Highlighting text
\begin{frame}
\frametitle{Sample frame title}

In this slide, some important text will be
\alert{highlighted} beause it's important.
Please, don't abuse it.

\begin{block}{Remark}
Sample text
\end{block}

\begin{alertblock}{Important theorem}
Sample text in red box
\end{alertblock}

\begin{examples}
Sample text in green box. "Examples" is fixed as block title.
\end{examples}
\end{frame}
%---------------------------------------------------------
\subsection{Computational physics}
\subsection{Experimental physics}

%---------------------------------------------------------
%Two columns
\begin{frame}
\frametitle{Two-column slide}

\begin{columns}

\column{0.5\textwidth}
This is a text in first column.
$$E=mc^2$$
\begin{itemize}
\item First item
\item Second item
\end{itemize}

\column{0.5\textwidth}
This text will be in the second column
and on a second tought this is a nice looking
layout in some cases.
\end{columns}
\end{frame}
%---------------------------------------------------------

%---------------------------------------------------------
%Thanks
\section{}
\begin{frame}%[plain]
	\begin{center}
	\Huge{\alert{Thanks for your attention!}}\par
	\Huge{\textit{Any questions?}}
    \end{center}
\end{frame}

\end{document}
