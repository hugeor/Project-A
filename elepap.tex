\documentclass[twocolumn,12pt,a4paper]{article}

\usepackage{xeCJK}
\usepackage{amsmath}
\usepackage{balance}
\usepackage{graphicx}
\usepackage{wrapfig}
\usepackage{subcaption}
\graphicspath{ {images/} }
\usepackage[backend=biber,
style=phys,]{biblatex}
\usepackage[top=1.25in,bottom=1.25in,left=1in,right=1in]{geometry}

\addbibresource{elex.bib}

\setCJKmainfont{simsun.ttf}
\setCJKsansfont{simhei.ttf}
\setCJKmonofont{simfang.ttf}

\setlength{\parindent}{2em}
\setlength{\parskip}{.5em}

\title{轴子暗物质与腔场探测}
\author{胡乃超\\(2013级物理学\, 12309011)}
\date{2014, 六月}

\begin{document}

\twocolumn[
\begin{@twocolumnfalse}
\maketitle
\renewcommand{\abstractname}{摘要}
\begin{abstract}
最近测定宇宙学参数表明我们的宇宙相当平坦,其总的能量密度大约由
$\frac{2}{3}$的真空能和$\frac{1}{3}$的非相对论性物质组成。我们所熟悉的
普通重子物质只是后者的一小部分。相对的,一种或多种暗物质因此被强烈认为
构成了宇宙中的物质的主要成分。轴子(axion)作为强 CP 问题Peccei—Quinn解中
出现的一种基本赝标量波色子,是暗物质的良好候选者。如果轴子存在,它必须
非常轻,质量范围约为$10^{-6}-10^{-3}eV$,并拥有极其微弱的对物质和辐射
的耦合。尽管如此,自 1983 年由 Sikivie 提出一种可行的基于Primakoff效应
的轴子搜索策略,目前的类似实验已经可以把系统噪音降至$\sim 100mK$甚至更
低。而日本的CARRACK 团队则重新设计了理论上能将有效噪音降至$\sim 12mK$
的新方案。有理由相信几年之内实验上便可以确定的探测或排除轴子。而正面的
探测结果同时也会提供轴子的精细结构,其中会包含大量的未知的有关银河系历
史、结构和动力学的信息。本文将介绍轴子的一些有用的理论特性,并简单介绍
探测轴子的两个重要实验的主要思路。\par
\emph{关键词:}轴子,冷暗物质,宇宙学,Primakoff效应
\end{abstract}
\end{@twocolumnfalse}
]

\renewcommand{\contentsname}{目录}
\tableofcontents

\section{引言}
近年来,依据宇宙学的发展和测量技术的精进
\cite{hanany00}\cite{perlmutter99},我们已经知道居住的宇宙是高度
平坦的($\Omega \sim 1$)。其中两个主要组成部分的分布约为真空能
($\Omega_{\Lambda} \sim 0.65$)和非相对论性物质($\Omega_m \sim 0.35$)。
而利用原初核合成(primordial nucleosynthesis)可分析出重子物质的相对密度
$\Omega \sim 0.045$\cite{schramm98}。\par
很多种观测包括星系团的动力学,单个星系的旋转曲线,轻元素的丰富程度,引
力透镜效应,宇宙大规模结构(LSS)和微波背景辐射的各向异性等均表明所搜寻
的暗物质必须是非重子的,非高热的和无碰撞的。具有所需性质的粒子被称为
``冷暗物质''(CDM)。最主要的CDM候选者包括大质量弱相互作用粒子(WIMPs)
和轴子两种。其中轴子是量子色动力学中著名的强CP问题的最吸引人的一个解,在
1977年由Peccei和Quinn首先提出。不过由于轴子的相对能量密度$\Omega_a$与
其质量$m_a$成反比,为了达到暗物质所占的相对能量密度,$m_a$的范围约在
$10^{-6}-10^{-3}eV$。这虽使得轴子的一些热性质与宇宙学观测结果吻合很好\cite{sikivie12},
却也对探测的仪器和方法提出极高要求。\par
尽管如此,Sikivie仍然建立起了一种非常优雅的探测星系轴子暗物质的方法
\cite{sikivie83a}\cite{sikivie85}。该方法利用Primakoff效应使轴子在调频
高Q腔场中转化为微波光子,从而进行有效探测。为证明其方案可行性,基于当
时的仪器便有实验进行。而这一方法至今更是主流轴子腔场探测方案的基础。另
一方面,对其进行较大修改并已取得非常好实验结果的里登堡原子单光子量子探
测实验也是很值得一提的。

\section{轴子理论回顾}
\subsection{强CP问题}
粒子物理中的标准模型(SM)在大型强子对撞机(LHC)发现了希格斯玻色子之后基
本很完善了。不用说,SM是一个成功的理论,如果不考虑强CP问题的话。
下面的规范对称性允许的量子色动力学拉氏量的四维散度项
\begin{equation}
  \mathcal{L} = \frac{\theta g_s^2}{32\pi^2}G_{\mu\nu}^a\tilde{G}^{a\mu\nu}
  \label{m:4-div}
\end{equation}
会破坏CP对称并贡献中
子电偶极矩(NEDM)。其中$g_s$是规范耦合常数,$G_{\mu\nu}^a$是胶子场强度,
$\theta$是一个常参数。而NEDM的实验数值范围
$|d_n|<2.9 \times 10^{-26} e\, cm$ \cite{baker06},这就要求
$\theta<0.7\times10^{-11}$\cite{kim10}。\par
那么问题就是$\theta$为什么这么小?因为从ABJ异常
\cite{adler69}\cite{bell69}可以证明式\ref{m:4-div}对$\theta$的依赖性可
以如下表示
\begin{equation}
  \bar{\theta} = \theta - arg \, det(m_1,m_2, \ldots, m_n),
  \label{m:depen}
\end{equation}
其中$m_i$为夸克质量。而夸克的质量起源自电弱理论,$\theta$则完全来自与
QCD理论。事实上SM完全没有提供任何理由让两个完全不同的物理吻合到如此程
度($10^{-11}$)。而这就是通常所说的强CP问题。\par
40多年前提出的轴子假设\cite{peccei77a}\cite{peccei77b}首先就是为了解决
这个强CP问题,而除此之外,其他的解决方式还有两种。第一种是取至少一种的
夸克质量为零。这会使得QCD对$\theta$的依赖性,这样CP便守恒。但是由于这
样会和现有的不止一种已经被验证的理论冲突,所以一般并不认为是合理的解。
而第二种是假定CP和(或)P对称性自发破缺。这种情况下\ref{m:depen}中
$\bar{\theta}$可以合理的被设置到很小的值。这种解法仍有很多研究在进行中。\par
而轴子解法则用了更为简单的方法,并在天体物理和宇宙学等方面有更多应用。
方法假定一种会自发破缺的$U_{PQ}(1)$拟对称性,即PQ对称性。然后考虑式\ref{m:depen}可
以做如下修正
\begin{multline}
  \bar{\theta} = \theta - arg \, det(m_1,m_2,\\ \ldots, m_n)- \frac{a(x)}{f_a},
  \label{m:modd}
\end{multline}
其中$a(x)$是轴子场,$f_a$叫做轴子衰变常数,代表对称破缺程度。此时非微
扰效应会产生一个对应的有效势能函数$V(\bar{\theta})$,其最小值在
$\bar{\theta}=0$处\cite{vefa84}。由此强CP问题得以解决。而解法提
出很快便意识到与PQ对称破缺相联系的是一种新的拟南部戈德斯通波色子
(quasi-Nambu-Goldstone boson),即轴子\cite{weinberg78}\cite{wilczek78}。
\subsection{轴子参数及一些限制}
\subsubsection{轴子参数}
轴子的特性参数用当前已有的代数技巧便可推出,此类文献也十分丰富,如
\cite{weinberg78}。不过由于轴子的模型不止一种,每种模型假设不同,推导
出的参数也可以有比较大的差异。最著名的两种是DFSZ模型
\cite{zhitnitsky80}\cite{dine81}和KSVZ模型\cite{kim79}\cite{shifman80},
但其中
多数参数选择也都不相同。不过本文重点只集中在轴子及其腔场探测实验的介绍
上,故不对质量之外的其他参数做过多解释。不过所有轴子的耦合均反比于
$f_a$。\par
而轴子质量$m_a$是实验探测的主要参数,则有比较统一认同的形式。一般认为轴
子从手性对称破缺中获得质量,其大小为
\begin{equation}
  \begin{split}
    m_a &= \frac{\sqrt{z}}{1+z}\frac{m_{\pi}f_{\pi}}{f_a} \\
        &\simeq 0.6 eV \frac{10^7GeV}{f_a},
  \end{split}
  \label{m:mass}
\end{equation}
其中$m_{\pi}$是$\pi$介子的质量,$f_{\pi}$是$\pi$介子的衰变常数,而
$z=m_u/m_d$则是上下夸克质量比,此处我们取标准值$0.56$。
\subsubsection{一些实验方面的限制}
重新考察与轴子质量关系密切的衰变常数$f_a$,其可以表示为
\begin{equation}
  f_a = \frac{v_a}{N},
  \label{m:fa}
\end{equation}
其中$v_a$称为真空期望值 (vacuum expectation value),会自发发破PQ对称性,
而$N$是表示PQ对称颜色异常 (color anomaly)的整数。由于$v_a$的自由性很高,
导致$m_a$的范围很大。实际上,模型提出最初计算的$m_a$比今天的最重假设还
要重$10^8$。本部分和下面一部分我们就介绍下对质量的主要限制条件。\par
首先,对于质量$m_a \geq 1MeV$,并迅速($10^{-11}$秒或更少)衰变成
$e^+e^-$的轴子模型的存在可能,已经由类似$\pi^+ \rightarrow a(e^+ e^-)+e^++\nu_e$
的衰变反应探测实验排除了\cite{eichler86}。第二,相对前面的模型,如果轴
子存活时间较长($10^{-11}$秒或更多),束晶堆方面的实验则对轴子质量有很强的限
制。由于实验中涉及较多过程,每种依赖不同的耦合。虽然难以精确计算总体效
果,却可以确定不会全部耦合恰好完全抵消。而已经进行了的很多这方面的探测\cite{pdg98}
已经把$m_a$的上限压缩至$\sim 50 keV$。下面考察一下天文观测对轴子质量提
出的限制。恒星燃烧的速率是由其释放能量效率限制的。而放射轴子这类极轻的
弱耦合波色子则是相当高效的能量释放方式。易见轴子的存在会加速恒星演化,
由此红巨星分支的丰富程度对轴子质量限制较强,如对KSVZ模型即可排除在$200keV
\geq m_a \geq 0.5eV$的范围\cite{raffelt87}。简单的说,对于红巨星
内部的热过程来说,重于$200keV$难以大量放射,而过轻的轴子则耦合很弱。
最后,$2eV \geq m_a \geq 3\times 10^{-3}eV$是由超新星SN1987A排除
的。这是由于观测到的中微子与理论上只有中微子有效冷却星核的推算基本一致。
\subsubsection{宇宙学方面的限制}
到这里为止,我们差不多进行的是对不断的对质量范围排除的介绍,而宇宙学的研究则提
供了一个指定的质量范围。首先,轴子会在PQ对称性自发破缺的时候发生一次相
变,一般称为PQ相变,这时其获得质量。相应的,轴子场也会开始振荡,之后转
化为不同形式的能量计入能量密度。那么由于对暴胀了解的缺乏,只能把以下的
讨论分为两种:$(1)$暴胀的再加热温度低于PQ相变温度,即暴胀发生在PQ相变
之后。此时轴子场相当于被暴胀同化,振荡会被以真空重整(vacuum
realignment)的形式计入能量密度。若用临界密度$\rho_c$表示即
\begin{equation}
  \label{m:case1}
  \begin{split}
    \Omega_a &= \frac{\rho_a(t_0)}{\rho_c} \\
    &=
    \frac{1}{6}\alpha^2(t_1)(\frac{f_a}{10^{12}GeV})^{7/6}(\frac{0.7}{h_{100}})^2,
  \end{split}
\end{equation}
其中$\alpha(t_1) = a(t_1)/f_a$称为初始偏差角,$t_1$为PQ相变的时间
\cite{asztalos01}。可以看出$\alpha(t_1)$有很大随机性,很难根据式
\ref{m:case1}推算$m_a$。不过幸运的是,形成今天的宇宙需要暗物质非完全均
匀,这使得下种情况更合理些。$(2)$暴胀的再加热温度高于PQ相变温度,即暴胀发
生在PQ相变之后。此时情况比上一种要复杂很多,很多文献详细描述了这种状况
下的物理。此处只取学界比较普遍承认的轴子转化能量密度,
\begin{equation}
  \label{m:case2}
  \begin{split}
    \Omega_a &= \Omega^{vac}+\Omega^{d.w.}+\Omega^{str}\\
&= (0.5-3.0)(\frac{f_a}{10^{12}})^{7/6}(\frac{0.7}{h_{100}})^2.
  \end{split}
\end{equation}
结合今天的$\Omega_a$数据和式\ref{m:mass}可以得到质量区间$1.3\mu eV <
m_a < 13\mu eV$\cite{asztalos01}。不过考虑到即使到今日宇宙学仍然不是完
全精确的科学,很多参数的不确定度依然很高,故可以认为合理区间约在
$1-100\mu eV$。\par
\begin{figure}
  \centering
  \includegraphics[width=0.46\textwidth]{range}
  \caption{$m_a$ ($f_a$)的范围}
  \label{fig:constr}
\end{figure}
图\ref{fig:constr}总结了到这里我们对$m_a$所讨论的全部内容。另外从实验
的角度出发,如前所讲,$\Omega_a$会随$m_a$的降低而增大,可行的实验方案
应从较低质量探测开始向质量较大方向推进。
\begin{figure}[h!]
  \centering
  \includegraphics[width=0.46\textwidth]{prima}
  \caption{轴子的Primakoff效应对应费曼图,其中$e$是电子,$I$表示离子。}
  \label{f:prima}
\end{figure}

\section{Sikivie实验}
\subsection{原理}
轴子可以通过刺激其在强磁场下转化为光子来探测
\cite{sikivie83a}\cite{sikivie85}。这是一种Primakoff效应,其费曼图如图
\ref{f:prima}。特别的,如果腔场的频率$\omega = 2\pi f$,轴子就会转化成
光子。\par
可以推倒出\cite{sikivie85}轴子转化为光子的功率是
\begin{equation}
  \label{m:pocov}
  \begin{split}
    P_a &= (\frac{\alpha}{\pi} \frac{g_\gamma}{f_a})^2 V B_0^2 \rho_a C \frac{1}{m_a} Min(Q_L,Q_a)\\
    &= 0.5\times10^{-26}W(\frac{V}{500 liter})(\frac{B_0}{7 T})^2 \\
    &\times  C (\frac{g_\gamma}{0.36})^2(\frac{\rho_a}{\frac{1}{2}\times10^{-24}g/cm^3})\\
    &\times (\frac{m_a}{2\pi(GHz)})Min(Q_L,Q_a),
\end{split}
\end{equation}
其中$V$是腔场容积, $Q_L$是腔洞的有载品质因子, $Q_a = 10^{16}$是银
晕轴子信号的$Q$因子,$\rho_a$是地面上的银晕轴子密度,$C$则是形状因子。
对于采用$TM_{010}$模式的圆柱形腔洞最大的$C$值约为$C \approx 0.69$。
\subsection{两个原理实验}

\section{CARRACK实验}
\subsection{Redberg原子}
\subsection{CARRACK I实验}

\section{总结}

\medskip
\balance
\renewcommand{\refname}{参考文献}
\printbibliography

\end{document}
