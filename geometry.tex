\documentclass[a4paper,11pt]{article}
\pdfoutput=1 % if your are submitting a pdflatex (i.e. if you have
             % images in pdf, png or jpg format)

\usepackage{jheppub} % for details on the use of the package, please
                     % see the JHEP-author-manual

\usepackage[T1]{fontenc} % if needed



\title{\boldmath A Brief Introduction to Geometry}


%% %simple case: 2 authors, same institution
%% \author{A. Uthor}
%% \author{and A. Nother Author}
%% \affiliation{Institution,\\Address, Country}

% more complex case: 4 authors, 3 institutions, 2 footnotes
\author{Naichao Hu}


% The "\note" macro will give a warning: "Ignoring empty anchor..."
% you can safely ignore it.

\affiliation{Department of Physics,\\Sun Yat-sen University}


% e-mail addresses: one for each author, in the same order as the authors
\emailAdd{hunch@mail2.sysu.edu.cn}





\begin{document} 
\maketitle
\flushbottom


\clearpage

\acknowledgments
These notes are far from original. They borrow heavily both from the books 
and the online resources listed described below. 
The first 
four 
were frequently consulted in the preparation of these notes, the next 
seven 
are other relativity texts which I have found to be consulted as further reinforcement, and the last
four 
are mathematical background references.
\begin{itemize}
\item Christopher Pope, \textit{Geometry and Group Theory}. A great
  text developing the basic notions of Manifolds and Geometry. Many
  elementary examples to help understanding new concepts. PDF can be
  found at:
  \url{http://people.physics.tamu.edu/pope/geom-group2006.pdf}
\item B.F. Schutz, \textit{A First Course in General Relativity}
  (Cambridge). This is a very nice introductory text. Especially
  useful if you are overwhelmed by the new definition of vectors and
  tangent vectors.
\item 
\end{itemize}

\clearpage

\section{Topological Spaces}
\label{s:topo}

\section{Manifolds}
\label{s:mani}

\section{Vector Fields}
\label{s:vefi}

%======================================================================
% \section{}
% \label{sec:intro}

% For internal references use label-refs: see section~\ref{sec:intro}.
% Bibliographic citations can be done with cite.
% When possible, align equations on the equal sign. The package
% \texttt{amsmath} is already loaded. See \eqref{eq:x}.
% \begin{equation}
% \label{eq:x}
% \begin{split}
% x &= 1 \,,
% \qquad
% y = 2 \,,
% \\
% z &= 3 \,.
% \end{split}
% \end{equation}
% Also, watch out for the punctuation at the end of the equations.


% If you want some equations without the tag (number), please use the available
% starred-environments. For example:
% \begin{equation*}
% x = 1
% \end{equation*}

% The amsmath package has many features. For example, you can use use
% \texttt{subequations} environment:
% \begin{subequations}\label{eq:y}
% \begin{align}
% \label{eq:y:1}
% a & = 1
% \\
% \label{eq:y:2}
% b & = 2
% \end{align}
% and it will continue to operate across the text also.
% \begin{equation}
% \label{eq:y:3}
% c = 3
% \end{equation}
% \end{subequations}
% The references will work as you'd expect: \eqref{eq:y:1},
% \eqref{eq:y:2} and \eqref{eq:y:3} are all part of \eqref{eq:y}.

% A similar solution is available for figures via the \texttt{subfigure}
% package (not loaded by default and not shown here). 
% All figures and tables should be referenced in the text and should be
% placed at the top of the page where they are first cited or in
% subsequent pages. Positioning them in the source file
% after the paragraph where you first reference them usually yield good
% results. See figure~\ref{fig:i} and table~\ref{tab:i}.

% \begin{figure}[tbp]
% \centering % \begin{center}/\end{center} takes some additional vertical space
% \includegraphics[width=.45\textwidth,trim=0 380 0 200,clip]{img1.pdf}
% \hfill
% \includegraphics[width=.45\textwidth,origin=c,angle=180]{img2.pdf}
% % "\includegraphics" is very powerful; the graphicx package is already loaded
% \caption{\label{fig:i} Always give a caption.}
% \end{figure}

% \begin{table}[tbp]
% \centering
% \begin{tabular}{|lr|c|}
% \hline
% x&y&x and y\\
% \hline 
% a & b & a and b\\
% 1 & 2 & 1 and 2\\
% $\alpha$ & $\beta$ & $\alpha$ and $\beta$\\
% \hline
% \end{tabular}
% \caption{\label{tab:i} We prefer to have borders around the tables.}
% \end{table}

% We discourage the use of inline figures (wrapfigure), as they may be
% difficult to position if the page layout changes.

% We suggest not to abbreviate: ``section'', ``appendix'', ``figure''
% and ``table'', but ``eq.'' and ``ref.'' are welcome. Also, please do
% not use \texttt{\textbackslash emph} or \texttt{\textbackslash it} for
% latin abbreviaitons: i.e., et al., e.g., vs., etc.




% \appendix
% \section{Some title}
% Please always give a title also for appendices.

% \paragraph{Note added.} This is also a good position for notes added
% after the paper has been written.


% BIBLIOGRAPHY
% use BIBTEX if you want
%\bibliographystyle{JHEP}
%\bibliography{yourBIBfiles}

% The bibliography will probably be heavily edited during typesetting.
% We'll parse it and, using the arxiv number or the journal data, will
% query inspire, trying to verify the data (this will probalby spot
% eventual typos) and retrive the document DOI and eventual errata.
% We however suggest to always provide author, title and journal data:
% in short all the informations that clearly identify a document.


% Please avoid comments such as "For a review'', "For some examples",
% "and references therein" or move them in the text. In general,
% please leave only references in the bibliography and move all
% accessory text in footnotes.

% Also, please have only one work for each \bibitem.


\end{document}
