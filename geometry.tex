\documentclass[a4paper,11pt]{article}
\pdfoutput=1 % if your are submitting a pdflatex (i.e. if you have
             % images in pdf, png or jpg format)

\usepackage{jheppub} % for details on the use of the package, please
                     % see the JHEP-author-manual

\usepackage[T1]{fontenc} % if needed

\usepackage{amsthm}
\theoremstyle{remark}
\newtheorem{remark}{Remark}[section]

\title{\boldmath Notes on Gravitation Part I: Geometry}


%% %simple case: 2 authors, same institution
%% \author{A. Uthor}
%% \author{and A. Nother Author}
%% \affiliation{Institution,\\Address, Country}

% more complex case: 4 authors, 3 institutions, 2 footnotes
\author{Naichao Hu}


% The "\note" macro will give a warning: "Ignoring empty anchor..."
% you can safely ignore it.

\affiliation{Department of Physics,\\Sun Yat-sen University}


% e-mail addresses: one for each author, in the same order as the authors
\emailAdd{hunch@mail2.sysu.edu.cn}





\begin{document} 
\maketitle
\flushbottom


\clearpage

\acknowledgments
These notes are far from original. They borrow heavily both from the books 
and the online resources listed described below. 
The first four were frequently consulted in the preparation of these
notes, the next six are other relativity texts which I have found to
be consulted as further reinforcement, and the last three 
are mathematical background references.
\begin{itemize}
\item Christopher Pope, \textit{Geometry and Group Theory}. A great
  text developing the basic notions of Manifolds and Geometry. Many
  elementary examples to help understanding new concepts. PDF can be
  found at:
  \url{http://people.physics.tamu.edu/pope/geom-group2006.pdf}
\item B.F. Schutz, \textit{A First Course in General Relativity}
  (Cambridge). This is a very nice introductory text. Especially
  useful if you are overwhelmed by the new definition of vectors and
  tangent vectors.
\item Sean M. Carroll, \textit{Lecture Notes on General
    Relativity}. These notes are lectures on introductory general
  relativity for beginning graduate students in physics and are
  basicly the draft of the later book, not very thorough on geometry
  though. PDF are available at:
  \url{http://xxx.lanl.gov/pdf/gr-qc/9712019v1}  
\item John C. Baez, Javier P. Muniain, \textit{Gauge Fields, Knots,
    and Gravity} (World Scientific). Slightly more mathematical, but
  very accessible.
\item Pankaj Sharan, \textit{Spacetime, Geometry and Gravitation}
  (Hindustan). A friendly text with three phases, demands very little
  prerequisites mathematically.
\item C. Misner, K. Thorne and J. Wheeler, \textit{Gravitation}
  (Freeman).A heavy book, in various senses. Second phase GR
  book. Everything thoroughly explained.
\item Eric Poisson, \textit{A Relativist's Toolkit: The Mathematics of
    Black-Hole Mechanics} (Cambridge). Second phase book with a brief
  look back at differential geometry and more attention on the
  application of GR on black holes.
\item R. Wald, \textit{General Relativity} (Chicago). Discussions of a
  number of advanced topics, including black holes, global structure,
  and spinors. The approach is more mathematically demanding than the
  previous books, and the basics are covered pretty quickly.
\item A.P. Lightman, W.H. Press, R.H. Price, and S.A. Teukolsky,
  \textit{Problem Book in Relativity and Gravitation} (Princeton). A
  sizeable collection of problems in all areas of GR, with fully
  worked solutions.
\item Gerard 't Hooft, \textit{Introduction to General Relativity}. A
  review of GR theory. Very brief thus rather demanding. PDF can be
  found at:
  \url{http://www.staff.science.uu.nl/~hooft101/lectures/genrel_2013.pdf}
\item Mikio Nakahara, \textit{Geometry, Topology and Physics} (Taylor
  \& Francis). A nice text introduces the ideas and techniques of
  differential geometry and topology at a level suitable for
  postgraduate students and researchers in these fields.
\item R.Aldrovandi, J.G.Pereira, \textit{An Introduction to
    Geometrical Physics} (World Scientific). Again a graduate level
  text. More physical and geometrical perspective, stressing the
  unifying power of the geometrical framework.
\item Yvonne Choquet-Bruhat, Cecile Dewitt-Morette, \textit{Analysis,
    Manifolds and Physics} (North Holland). A reference book, which
  has found wide use as a text, provides an answer to the needs of
  graduate physical mathematics students and their teachers. More
  rigorous than previous works.
\end{itemize}

\clearpage

\section{Topological Spaces}
\label{s:topo}
Before being able to define a manifold, we need to introduce the
notion of a topological space. This can be defined after a topology is defined.

Let $X$ be a set and $T$ be a family of subsets of $X$. Elements of $T$
are called open sets. And $T$ is called a \textbf{topology} on X if: 
\begin{itemize}
\item The union of any number of open subsets is an open set.
\item The intersection of a finite number of open subsets is an open set.
\item Both $X$ itself, and the empty set $\emptyset$, are open.
\end{itemize}
Set $X$ with a given topology defined on it, i.e. the pair $(X,T)$, is a
\textbf{topological space}. An open set containing a point $x\in X$ is
called a \textbf{neighbourhood} of $x$. The complement of an open set is \textbf{closed}.
\par
The use of a topology is that it allows us to define continuous
functions. A function $f: X \mapsto Y$ from one topological space to
another is defined to be \textbf{continuous} if, given any open set $U
\subseteq Y$, the inverse image $f^{-1}U \subseteq X$ is open. (The
above definition is equivalent to the epsilon-delta definition.)\par
The idea of a manifold is that, like a globe, we can cover it with
patches that look like just like $\mathbb{R}^n$. A \textbf{cover} of a set $A$
in a topological space $(X,T)$ is a family of sets ${U_\alpha}$ such
that $A\subset\cup_{\alpha}U_{\alpha}$. A cover ${U_{\alpha}}$ of $A$
in a topological space $(X,T)$ is called open if all $U_{\alpha}$ are
open in $(X,T)$. A family of sets ${V_{\alpha}}$ is called a subcover
of $A$ in $(X,T)$ if
\begin{itemize}
\item ${V_{\alpha}}$ is a subset of the cover ${U_{\alpha}}$ of $A$;
\item ${V_{\alpha}}$ is a cover of $A$.
\end{itemize}
A topological space $(X,T)$ is said to be \textbf{compact} if every open
cover of $(X,T)$ has a finite subcover.\par
Some further concepts need to be introduced. First, we define a basis
for the topology of $X$ as some subset of all possible open sets in
$T$, such that by taking intersections and unions of the members of
the subset, we can generate all possible open subsets in $T$.\par A
topological space $(X,T)$ said to satisfy the \textbf{First axiom of
  separation} (or to be $\mathbf{T_1}$ \textbf{space}) if for all two
distinct points $x$ and $y$ in $(X,T)$ there exists an open
neighborhood $O_x$ of the point $x$ such that $y\not\in O_x$ and there
exists a open neighbourhood $O_y$ of the point y such that $x\not\in
O_y$.\par And we now may define a Hausdorff Space. A topological space
$(X,T)$ is a \textbf{Hausdorff space} (or a $\mathbf{T_2}$\textbf{-space} or
a separated space) if all two distinct points in $X$ have two disjoint
neighbourhoods. The great mathematician and geometer Michael Atiyah gave a nice
colloquial definition: ``A topological space is Hausdorff if the points
can be housed off.'' 
\section{Manifolds}
\label{s:mani}
We can now give a formal definition of a \textbf{smooth $n$-manifold}
$M^n$, with a smooth atlas of charts, as
\begin{itemize}
\item A topological space $(X, T)$ with covers
\item A collection (called \textbf{atlas}) of maps $\phi_i: U_i
  \mapsto \mathbb{R}^n$ called \textbf{charts}, which define a $1-1$
  relation between points in $U_i$ and points in an open ball in
  $\mathbb{R}^n$ such that the \textbf{transition function}
  $\phi_{\alpha}\circ \phi^{-1}_{\beta}$ is smooth where it is
  defined.
\end{itemize}
\begin{remark}The idea is to divide the whole $n$-manifold into
  patches that looks like $\mathbb{R}^n$. The map $\phi_i$ that
  connects both is the notion of choosing coordinates on the patch
  $U_i$.
\end{remark}
\begin{remark}
  We are now allowed to tell if a function on $M^n$ is smooth using
  charts, since we defined a smooth manifold. (A \textbf{topological
    manifold} only require its transition function to be continuous.)
  A function is said to be $smooth$ if $\forall \alpha,
  f\circ\phi_{\alpha}^{-1}: \mathbb{R}^n\mapsto\mathbb{R}$ is
  smooth. Also, we will always assume our manifold are ``Hausdorff''
  and ``paracompact''.%to be explained
\end{remark}
\begin{remark}
  Consider points lie in the intersection $V$ of two patches and let
  $\phi_1$ and $\phi_2$ be the two charts defined respectively on the
  patches. We are therefore enabled to go back and forth between the
  two copies of $\mathbb{R}^n$. Suppose that we choose coordinates
  $x^i$ on the first copy, and coordinates $\tilde{x}^i$ on the second
  copy. Thus we obtain two function: $x^i =
  \phi_1\circ\phi_2^{-1}(\tilde{x}^j)$ and $\tilde{x}^i =
  \phi_2\circ\phi_1^{-1}(x^j)$. When a neighbourhood instead of a
  point is considered, we bound to ask the question of whether the
  functions are smooth or not. Thus we are led to the notion of a
  \textbf{differentiable manifold}, as being a manifold where the
  coordinates covering any pair of overlapping patches are smooth,
  differentiable functions of one another. And in practice, one tend
  to assume everything is $C^{\infty}$ differentiable.
\end{remark}\par
Two atlases are said to be \textbf{compatible} if, wherever there are
overlaps, the transition functions are smooth. It is not necessarily
the case that the charts in one atlas are compatible with the charts
in another atlas.\par
One last concept on manifold to introduce is orientable manifolds. A
manifold is said to be orientable if it admits an atlas such that in
all overlapping regions between charts, the Jacobian of the relation
between the coordinate systems satisfies$$det\left( \frac{\partial
    x^i}{\partial\tilde{x}^j} \right)>0.$$
\section{Vectors and Tangent Vectors}
\label{s:VT}
We now turn to a discussion of vectors and tensors on manifolds. We
should begin this discussion by forgetting certain things about
vectors that we learned in kindergarten. The vector concept was
introduced with the notion of position vectors. All is good as long as
one is only going to describe Euclidean space with Cartesian
coordinates, but it's not a valid way defining a vector in general,
like in a curved space, or even in a flat space equipped with
non-cartesian coordinates.
\subsection{Vector Fields}
\label{sub:vefi}
To introduce a new definition of vectors, we need some preparation
first. Let's define vector fields on a manifold $M$ first. The key to
defining vector fields on a manifold is o note that given a field of
arrows, one can take directional derivative in the direction of the
arrows. Let $C^{\infty}(M)$ be the set of smooth (real-valued)
functions on a manifold $M$. The following batch of rules holds trivially:
\begin{align*}
  f +g &= g+f\\
  f+(g+h)&=(f+g)+h\\
  f(g h)&=(f g)h\\
  f(g+h)&=f g+f h\\
  (f+g)h&=f h+g h\\
  1 f &= f\\
  \alpha( \beta f) &= ( \alpha \beta ) f \\
  \alpha(f+g) &= \alpha f+ \alpha g \\
  ( \alpha + \beta )f &= \alpha f + \beta f \\
  f g &=g f,
\end{align*}
where $f, g, h \in C^{\infty}(M)$ and $\alpha, \beta \in \mathbb{R}$.
We thus know that $C^{\infty}(M)$ is a commutative algebra over the
real numbers. And a \textbf{vector field} $v$ is defined to be a
function from $C^{\infty}(M)$ to $C^{\infty}(M)$ satisfying the
following properties:
\begin{align*}
  v(f+g)&=v(f)+v(g)\\
  v(\alpha f)&=\alpha v(f)\\
  v(f g)&=v(f) g+f v(g),
\end{align*}
$\forall f, g \in C^{\infty}(M)$ and $\alpha \in \mathbb{R}$. Here we
have isolated all the basic rules a directional derivative operator
should satisfy. And we have achieved this relying on no choice of
coordinates on $M$!\par
Let $Vect(M)$ be the collection of all vector fields on $M$. Given $v,
w \in Vect(M), f, g \in C^{\infty}(M)$, we define $v+w$ by $$(v+w)(f) =
v(f) + w(f),$$ and $g v$ by $$(g v)(f) = g v(f).$$ Then when the
following rules:
\begin{align*}
  f(v + w) &= f v + f w\\
  (f + g)v &= f v + g v\\
  (f g)v &= f(g v)\\
  1 v &= v,
\end{align*}
(here ``1'' denotes the constant function equal to 1 on all of $M$),
we say that $Vect(M)$ is a \textbf{module over} $C^{\infty}(M)$.\par
Consider the example of the vector fields ${\partial_\mu}$ on
$\mathbb{R}^n$ that span $Vect(\mathbb{R}^n)$ as a module over
$C^{\infty}(M)$. Every vector field on $\mathbb{R}^n$ is a linear
combination of the form $$v = v^{\mu} \partial_{\mu},$$ for some
function $v^{\mu} \in C^{\infty}(\mathbb{R}^n)$. We can readily see
${\partial_{\mu}}$ on $\mathbb{R}^n$ are linearly independent, which
implies every field $v$ on $\mathbb{R}^n$ has a unique representation
as a linear combination $v^{\mu} \partial_{\mu}$. Thus the vector
fields ${\partial_{\mu}}$ form a \textbf{basis} of
$Vect(\mathbb{R}^n)$. The functions $v^{\mu}$ are called the
\textbf{components} of the vector field $v$.
\subsection{Tangent Vectors}
\label{sub:TV}
A tangent vector should let us take the directional derivative at the
point. Say given a vector field $v$ on $M$, we can take the derivative
$v(f)$ and evaluate at the point $p$, for all $f \in
C^{\infty}(M)$. Indeed, we can define $$v_p: C^{\infty}(M)\mapsto
\mathbb{R}$$ by $$v_p(f)=v(f)(p).$$ And $v_p$ is the \textbf{tangent vector} at
$p$. Following from section~\ref{sub:vefi}, $v_p$ has three basic properties:
\begin{align*}
  v_p (f+g) &= v_p(f) + v_p(g) \\
  v_p(\alpha f) &= \alpha v_p(f) \\
  v_p(f g) &= v_p(f)g(p) + f(p)v_p(g).
\end{align*}
Let $T_pM$, the \textbf{tangent space} at $p$, denote the set of all
tangent vectors at $p\in M$. \par
\begin{remark}
  $\forall p\in M$, a vector field $v\in Vect(M)$ determines a tangent
  vector $v_p\in T_pM$. Also, every tangent vector at $p$ is of the
  form $v_p$ for some vector field or other.\par
\end{remark}
\begin{remark}
  The dimension of $T_pM$ is $n$, the dimension of the manifold
  $M$, since ${\partial_{\mu}}$ is a basis. \par
\end{remark}
Now consider a patch $U$ in the manifold $M^n$, for which a local
coordinates $x^{\mu}$ or $x^{\nu '}$ is introduced. We can use the
chain rule to convert between the two bases:$$v =
v^{\mu}\partial_{\mu} = v^{\mu}\frac{\partial x^{\nu '}}{\partial
  x^{\mu}}\partial_{\nu}\equiv v^{\nu '}\partial_{\nu '}.$$
And we thus obtain the rule$$v^{\nu '} = v^{\mu}\frac{\partial x^{\nu '}}{\partial
  x^{\mu}},$$ which is called \textbf{general coordinate
  transformation}.\par





\section{$\ldots$}
%======================================================================
% \section{}
% \label{sec:intro}

% For internal references use label-refs: see section~\ref{sec:intro}.
% Bibliographic citations can be done with cite.
% When possible, align equations on the equal sign. The package
% \texttt{amsmath} is already loaded. See \eqref{eq:x}.
% \begin{equation}
% \label{eq:x}
% \begin{split}
% x &= 1 \,,
% \qquad
% y = 2 \,,
% \\
% z &= 3 \,.
% \end{split}
% \end{equation}
% Also, watch out for the punctuation at the end of the equations.


% If you want some equations without the tag (number), please use the available
% starred-environments. For example:
% \begin{equation*}
% x = 1
% \end{equation*}

% The amsmath package has many features. For example, you can use use
% \texttt{subequations} environment:
% \begin{subequations}\label{eq:y}
% \begin{align}
% \label{eq:y:1}
% a & = 1
% \\
% \label{eq:y:2}
% b & = 2
% \end{align}
% and it will continue to operate across the text also.
% \begin{equation}
% \label{eq:y:3}
% c = 3
% \end{equation}
% \end{subequations}
% The references will work as you'd expect: \eqref{eq:y:1},
% \eqref{eq:y:2} and \eqref{eq:y:3} are all part of \eqref{eq:y}.

% A similar solution is available for figures via the \texttt{subfigure}
% package (not loaded by default and not shown here). 
% All figures and tables should be referenced in the text and should be
% placed at the top of the page where they are first cited or in
% subsequent pages. Positioning them in the source file
% after the paragraph where you first reference them usually yield good
% results. See figure~\ref{fig:i} and table~\ref{tab:i}.

% \begin{figure}[tbp]
% \centering % \begin{center}/\end{center} takes some additional vertical space
% \includegraphics[width=.45\textwidth,trim=0 380 0 200,clip]{img1.pdf}
% \hfill
% \includegraphics[width=.45\textwidth,origin=c,angle=180]{img2.pdf}
% % "\includegraphics" is very powerful; the graphicx package is already loaded
% \caption{\label{fig:i} Always give a caption.}
% \end{figure}

% \begin{table}[tbp]
% \centering
% \begin{tabular}{|lr|c|}
% \hline
% x&y&x and y\\
% \hline 
% a & b & a and b\\
% 1 & 2 & 1 and 2\\
% $\alpha$ & $\beta$ & $\alpha$ and $\beta$\\
% \hline
% \end{tabular}
% \caption{\label{tab:i} We prefer to have borders around the tables.}
% \end{table}

% We discourage the use of inline figures (wrapfigure), as they may be
% difficult to position if the page layout changes.

% We suggest not to abbreviate: ``section'', ``appendix'', ``figure''
% and ``table'', but ``eq.'' and ``ref.'' are welcome. Also, please do
% not use \texttt{\textbackslash emph} or \texttt{\textbackslash it} for
% latin abbreviaitons: i.e., et al., e.g., vs., etc.




% \appendix
% \section{Some title}
% Please always give a title also for appendices.

% \paragraph{Note added.} This is also a good position for notes added
% after the paper has been written.


% BIBLIOGRAPHY
% use BIBTEX if you want
%\bibliographystyle{JHEP}
%\bibliography{yourBIBfiles}

% The bibliography will probably be heavily edited during typesetting.
% We'll parse it and, using the arxiv number or the journal data, will
% query inspire, trying to verify the data (this will probalby spot
% eventual typos) and retrive the document DOI and eventual errata.
% We however suggest to always provide author, title and journal data:
% in short all the informations that clearly identify a document.


% Please avoid comments such as "For a review'', "For some examples",
% "and references therein" or move them in the text. In general,
% please leave only references in the bibliography and move all
% accessory text in footnotes.

% Also, please have only one work for each \bibitem.


\end{document}
