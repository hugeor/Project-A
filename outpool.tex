\documentclass[10pt,a4paper]{article}

\usepackage{xeCJK}
\usepackage{amsmath}
\usepackage{balance}
\usepackage{setspace}
\usepackage[titletoc]{appendix}
\usepackage{minted}
\usepackage{graphicx}
\usepackage{wrapfig}
\usepackage{subcaption}
\graphicspath{ {images/} }
%\usepackage[backend=biber,
%style=phys,]{biblatex}

\usepackage[top=1.25in,bottom=1.25in,left=1in,right=1in]{geometry}
\usepackage{indentfirst} 
%\addbibresource{elex.bib}


\setCJKmainfont{simsun.ttf}
\setCJKsansfont{simhei.ttf}
\setCJKmonofont{simfang.ttf}

\setlength{\parindent}{2em}
\setlength{\parskip}{.5em}

\renewcommand{\figurename}{图}
\renewcommand{\tablename}{表}

\title{关于2013级“基础学科拔尖学生培养实验班”拟入选学生名单的异议}

\author{胡乃超}
\date{}

\begin{document}
\begin{spacing}{1.6}
\maketitle
\section{引言}
学生以为录取名单会在理工院首页有公示,不过今早偶然打开逸仙学院首页时才知道结果已然出来几天了。很遗憾自己现在并没有出现在拟入选的学生名单之列,几个友人的录取倒在意料之中。仔细想来,大约是做了一个太差的个人展示,比较没有重点,加上个人的一点怯场和疲倦才导致了现在的结果。不过还是很相信专家组对学生们的判断能力,毕竟之前转专业面试时紧张的思路都不连续的我还是被陈敏院长通过。又想起陈老师在宣讲会当天有讲落选的学生仍然有申诉的机会,学生于是再次呈现一份文字版的个人展示。借此希望老师们可以给与一个重新考虑学生的申请的机会。\par

\section{时间顺序上对物理学的已有探索}
由于学生在面试名单中课程成绩并无所突出,以下陈述仍多为课外探索。与面试时相比,会多增些细节。
\subsection{2013年暑假及之前}
基础科学对我的吸引一直都在,这吸引很像在新雪上踩下脚印的诱惑。2013年春季学期时开始读一些科普书籍,总会或多或少的和暗物质有些关系。而那时又恰好Higgs粒子热,于是学生通过转专业面试之后暑假在准备申请本科生科研项目时就读起了Griffiths的粒子物理书\cite{grif-ep}。这时是第一次读一些和课程完全无关的书,也是第一次发现其实读一些面向本科生写的前沿话题的书对了解整个学科的益处。而且
\subsection{2013-2014学年第二学期}
\subsection{2014年寒假及2013-2014学年第三学期}
\subsection{2014暑假至今}

\section{下面一年对物理学方面的计划}
\section{写在最后}
最后写一些不好分类到其他部分的东西。\par
\begin{table}[h!]
\centering
\caption{成绩分布与学分总览}
\begin{tabular}{c c c}
\hline
学分类型 & 绩点 & 学分\\ [.2ex]
\hline \hline
公必 & 3.33	&21/36  \\
专必 & 3.88	&21/66 \\
专选 & 4.12 &13/32 \\
公选 & 3.1	&23/16 \\
总览 & 3.73	&78/150 \\ [.2ex]
\hline
\end{tabular}
\label{t:gr}
\end{table}
首先重新回答一下陈老师在面试的时候问我的绩点问题。因为虽然当时的回答并没有不对,不过似乎容易给人误解。学生两年的总绩点汇总如表\ref{t:gr}\footnote{总览项的绩点是除去公选之外的绩点,算法与教务系统相同。因教务系统目前显示新学期未注册,故数据由第三方教务系统提供。}所示。很明显排名靠后的主要原因是由于公共必修课。\par
关于逸仙学院入学条件。

\end{spacing}

\medskip
\renewcommand\refname{提到的书籍}
%\printbibliography %print出来就是references
\bibliographystyle{unsrt}
\bibliography{bib}
\end{document}
