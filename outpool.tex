\documentclass[10pt,a4paper]{article}

\usepackage{xeCJK}
\usepackage{amsmath}
\usepackage{balance}
\usepackage{setspace}
\usepackage[titletoc]{appendix}
\usepackage{minted}
\usepackage{url}
\usepackage{graphicx}
\usepackage{wrapfig}
\usepackage{subcaption}
\graphicspath{ {images/} }
%\usepackage[backend=biber,
%style=phys,]{biblatex}

\usepackage[top=1.25in,bottom=1.25in,left=1in,right=1in]{geometry}
\usepackage{indentfirst} 
%\addbibresource{elex.bib}


\setCJKmainfont{simsun.ttf}
\setCJKsansfont{simhei.ttf}
\setCJKmonofont{simfang.ttf}

\setlength{\parindent}{2em}
\setlength{\parskip}{.5em}

\renewcommand{\figurename}{图}
\renewcommand{\tablename}{表}

\title{关于2013级“基础学科拔尖学生培养实验班”拟入选学生名单的异议}

\author{胡乃超}
\date{}

\begin{document}
\begin{spacing}{1.6}
\maketitle
\section{引言}
学生以为录取名单会在理工院首页有公示,不过今早偶然打开逸仙学院首页时才知道结果已然出来几天了。
很遗憾自己现在并没有出现在拟入选的学生名单之列,几个友人的录取倒在意料之中。
仔细想来,大约是做了一个太差的个人展示,比较没有重点,加上个人的一点怯场和疲倦才导致了现在的结果。
不过还是很相信专家组对学生们的判断能力,毕竟之前转专业面试时紧张的思路都不连续的我还是被陈敏院长通过。
又想起陈老师在宣讲会当天有讲落选的学生仍然有申诉的机会,学生于是再次呈现一份文字版的个人展示。
借此希望老师们可以重新考虑下学生的申请。\par

\section{从时间顺序上论述对物理学的已有探索}
由于学生在面试名单中课程成绩并无所突出,以下陈述仍多为课外探索。与面试时相比,会多增些细节。
\subsection{2013年暑假及之前}
基础科学对我的吸引一直都在,这吸引很像在新雪上踩下脚印的诱惑。
2013年春季学期时开始读一些科普书籍,总会或多或少的和暗物质有些关系。
而那时又恰好Higgs粒子热,于是学生通过转专业面试之后暑假在准备申请本科生科研项目
\footnote{项目名称:轴子暗物质探测方案研究}时就读起了
Griffiths的粒子物理书\cite{grif-ep}。虽然那个暑假只读到一些类似科普汇总的部分,仍是整个暑假很兴奋的样子。
这时是第一次读一些和课程完全无关的书,也是第一次发现其实读一些面向本科生写的前沿话题的书对了解整个学科的益处。
而且读到不懂的地方时常需要很多搜索工作,加上对一些完全没接触的学科(比如量子力学)不问原理的直接应用,对能力也是种锻炼。
\subsection{2013-2014学年第二学期}
这是真正到物理系就读的第一个学期。这学期犯了比较多的错误,走过的弯路很长。\par
这时对降级这件事不以为然,数学基础与12级类似的学生开始跑去蹭四大力学的课。
不过约一个多月之后,学生觉得理论力学和统计力学的课程的进度都并不很快,就开始了自学。
选择的两门力学是经典力学和量子力学。其中量子力学还好,教材选了Griffiths的\cite{grif-qm},难度适中。
而经典力学的主要教材则失败的选择了研究生难度的Goldstein\cite{gold-cm}。
这个失败的选择对应的后果就是两门力学都只学了一部分,而其中花掉大部分时间的经典力学甚至只读了第一章,而且并未完成所有习题。
\par
而10月份开始,得到项目导师对之前准备工作的认可之后,学生还开始断断续续的花费精力去阅读关于轴子的文献,
以期在非数学严格的层面理解到轴子的起源及一些基本的性质。到12月项目正式批准下来,所读的文献就开始变成了大型探测实验为主,
不过这时还只是停在了解阶段。\par
这学期实际上还夹杂一些对统计力学的探索,不过很浅就不提了。
总的来说基本接触的新方向都停留在虽然不能解决该方向的复杂问题,不过还算达到有学习相关问题的基础
(比如刚接触非相对论性弦的运动时还是优先使用了拉氏量解决)。
不足之处也非常明显,比如即使花费大量精力,仍没有一门力学算是打好基础。

\subsection{2014年寒假及2013-2014学年第三学期}
到寒假时主要的学习时间实际上到开学前两周左右才开始,那时主要在准备或在学coursera上的三门课程:
Galaxies and Cosmology (Caltech),  
An Introduction to Functional Analysis (ECOLE  CENTRALE  PARIS),  
Statistical Mechanics: Algorithms and Computations (ÉNS)。 
申请时交的三张证书也就是这三门课的了。三门课几乎在同一时间开始,持续时间大概在8到10周左右
(即在第三学期期中左右时间结束)。其中Caltech的课是同时他本校的大二学生也在上的课程,
要求至少一门天文基础课,而学生修习的时候是第一次接受宇宙学的知识。
好在指定的两本教材之一Ryden\cite{ryde-co}写的非常基础,后来才坚持学完这门课程。
不过大概由于教材简单,课程的进度其实是非常快的,大概五周不到的时间
里教授的阅读任务就布置完了这本教材和Schneider\cite{schn-ea}的一部分。
而第二门ECP的泛函课相比来说就容易很多。虽然老师编了很详细的笔记,不过由于同时修了三门课,学生并没仔细读那些。
而课程的设定也是不读笔记的话,只能理解课程的一部分,只是这部分也超过拿证书的分数。
目前学生在学广相里面的微分几何还是常常用到里面的一些概念的。
至于ENS开设的统计力学课,是由物理系主任主讲的。
物理内容在研究生的难度,里面包含了比如费曼路径积分,波色-爱因斯坦凝聚和磁体内的自旋方向等高等内容。
加上本来学生对统计力学就了解不足,很难说从其中学到物理思想。
不过由于课程采用教授介绍思路,助教讲授数学细节的形式。
学生应对这门课的主要方法变成听过思路之后直接开始编程解决问题,然后再和助教的讲解比对。
最终的效果就是编程能力和逻辑能力有一些提高了。\par
三门课程加上学院的课程几乎占用了前半学期的全部时间。而这个学期后半部分的时间学生的工作主要包括:研读更多轴子的综述,
对轴子的理解做一些总结\footnote{本为电磁学课程要求的前沿问题综述,学生刚好就整理了一下轴子的相关文献。总结见附件-轴子}
以及结合电磁学的课程自学电动力学\cite{grif-em}。\par
这学期总结来说虽然成绩方面不如前一学期,不过对物理的探索方面有一些进步。花费的精力在之后的学习中也开始显示出作用。
比较明显的不足是学习过程中很少和人讨论,对一些问题的思路没有打开。
物理学社的同学比较多机会锻炼到的,做基础物理很需要的作报告的能力学生却完全没有锻炼过。
\subsection{2014暑假至今}
暑假本来的计划是开始读过一篇老师给的关于量子信息的文章,开始思考轴子的新探测器。
不过开始读纯实验方面的文献时发现实验家也有很多专门的术语来描述实验。而与理论不同的是,这些术语多数从不会在教科书中出现。
后来8月时,在网上随便搜索准备换换脑子的时候,开始发现面向本科生的蛮有意思的东西,
包括t' Hooft的给物理学生的书单\cite{hoof-ph}和像Zwiebach\cite{zwie-st}。
因为一直阅读的轴子的文献经常不加说明的插入弦论的计算,学生便从Zwiebach的书开始读起。
到面试前的时候,完整读了相对论性电磁学,空间卷曲和非相对论性弦的运动几个部分。
到那里为止还没有出现数学或物理工具的不够用。

\section{下面一年对物理学方面的计划}
在轴子方面,决定在正式开学之后开始仔细的请教实验方面的知识,尽可能在剩下几个月的时间里构建出轴子探测器的新想法,
或者一点对已有方案的改进。\par
由于暑假期间也开始利用Mukhanov的书\cite{mukh-pf}学习一些进阶宇宙学。不过由于这本书已经需要广义相对论的基础,所以
开始结合利用Schwarz\cite{schw-sr}和Schutz\cite{schu-gr}系统学习相对论,相关的几何参考书不在此列举。\par
关于量子引力方面,学生打算再继续学一些弦论基础之后,开始用与\cite{zwie-st}很接近的本科生教材\cite{gamb-lg}来了解一下
圈量子引力方面的知识。

\section{写在最后}
\subsection{关于面试内容的一点解释}
\begin{table}[h]
\centering
\caption{成绩分布与学分总览}
\begin{tabular}{c c c}
\hline
学分类型 & 绩点 & 学分\\ [.2ex]
\hline \hline
公必 & 3.33	&21/36  \\
专必 & 3.88	&21/66 \\
专选 & 4.12 &13/32 \\
公选 & 3.1	&23/16 \\
总览 & 3.73	&78/150 \\ [.2ex]
\hline
\end{tabular}
\label{t:gr}
\end{table}
首先重新回答一下陈老师在面试的时候问我的绩点问题。因为虽然当时的回答并没有不对,不过似乎容易给人误解。
学生两年的总绩点汇总如表\ref{t:gr}\footnote{总览项的绩点是除去公选之外的绩点,算法与教务系统相同。
因教务系统目前显示新学期未注册,故数据由第三方教务系统提供。}所示。很明显排名靠后的主要原因是由于公共必修课。\par
其次,关于逸仙学院入学条件。学生面试回来就搜索了一下,仍然没有理解陈老师所指的是什么。
不过在学生理解来说,我是一个喜欢基础研究,希望最后以科研做工作的人;我也没有知识停留在这样说说,然后只接受外界提供的
条件正常学习,而是在尽自己所能的把科研融进生活。那么我觉得我是符合进入条件的,剩下的判断工作就要交给专家组了,
而我对专家组的判断力也是信任的。\par
再次,关于从没有联系中大的教授。准确来说我知道的比较符合我目前心仪方向的是李淼教授,不过蛮可惜的是学生并没有找到李老师
公开在中大这边的联系方式。而且目前学生产生的问题多数只是学习中的问题,而不是研究中的问题,直接联系教授似乎会有所打扰。
另外学生也是生性腼腆,多数产生的问题会自己搜索得到,或者在网\cite{phyfor}上发贴求助。当然学生也意识到这样的问题,这也是
学生如此想进逸仙学院的原因之一。\par
然后,关于数学建模获奖及高师班的问题。学生两个问题倒不是没有自信直接回答。
只是或许性格使然,总觉得没有任何依据的对数模获奖进行预测并不科学。
而高师班的问题是因为学生知道之前华梦师姐并没有进高师班,原因是导师并不认为高师班对她的发展最好。
学生本身自然想进入学习,毕竟从高师的数学名气和之前上课的经验来说高师班都是极好的训练物理的地方。
不过学生却并不觉得对这些事的理解会深于导师(如果有机会有的话),所以回答并不去确定。\par
\subsection{最后之后}
到这里,学生已经尽可能的吧所有之前的努力和失败一并呈现出来。
虽然不知道别人的情况,但若略微自负一次,学生自觉思维能力和努力情形并不输于申请的任何人;对逸仙学院所提供的资源似乎也
不会较别人有任何浪费。故特别请求老师们可以重新审视学生的申请。如果如何并不能在今年进入学院学习,也恳请老师能对学生申请
的缺陷和明年的努力方向做一些指点。对于给您们造成的麻烦,学生深表歉意。

\end{spacing}

\medskip
\renewcommand\refname{提到的书籍和网页}
%\printbibliography %print出来就是references
\bibliographystyle{unsrt}
\bibliography{bib}
\end{document}
