\documentclass{beamer}
\usepackage[utf8]{inputenc}

\usepackage{utopia} %font utopia imported
\usefonttheme{professionalfonts}
\usepackage{graphicx}
\usepackage{wrapfig}

\graphicspath{ {images/} }

\usetheme{Antibes}
\usecolortheme{beaver}

\usepackage{xeCJK}
\setCJKmainfont{simsun.ttf}
\setCJKsansfont{simhei.ttf}
\setCJKmonofont{simfang.ttf}

%------------------------------------------------------------
%This block of code defines the information to appear in the
%Title page
\title[About] %optional
{The Road to Reality}

\subtitle{A Brief Guide to Active Endeavors and More}

\author[Hunc]{胡乃超}
\institute[SPE]{Department of Physics\\
School of Physics and Engineering}


\date[SYSU 2014] % (optional)
{引力波楼, Sep 2014}

%\logo{\includegraphics[height=1.5cm]{lion-logo.png}}

%End of title page configuration block
%------------------------------------------------------------



%------------------------------------------------------------
%The next block of commands puts the table of contents at the 
%beginning of each section and highlights the current section:

%\AtBeginSection[]
%{
%  \begin{frame}
%    \frametitle{Table of Contents}
%    \tableofcontents[currentsection]
%  \end{frame}
%}
%------------------------------------------------------------


\begin{document}

%The next statement creates the title page.
\frame{\titlepage}


%---------------------------------------------------------
%This block of code is for the table of contents after
%the title page
%\begin{frame}
%  \frametitle{Table of Contents}
%  \tableofcontents
%\end{frame}
%---------------------------------------------------------

%=========================================================
\section{In pursuit of the most challenging}

%---------------------------------------------------------
\begin{frame}
  \frametitle{Several expeditions undertaken in physics}
  Here I list some of most advanced physics I can think of, many of
  which I am trying to get a feel for. 
  \begin{itemize}
  \item<2-> Theoretical physics 
    \begin{itemize}
    \item Cosmology
    \item Particle physics
    \item String theory
    \end{itemize}
  \item<3-> Computational physics 
    \begin{itemize}
    \item The Monte Carlo method
    \end{itemize}
  \item<4-> Experimental physics 
    \begin{itemize}
    \item Axions
    \end{itemize}
  \end{itemize} 
    
\end{frame}

%---------------------------------------------------------

%---------------------------------------------------------
\begin{frame}
  \frametitle{Problems hard to tackle alone}
  \begin{itemize}
  \item<1-> One thing about advanced topics is that they require good
    resources to obtain fair understanding. And without help from the
    experienced, the progress of seeking the right way can be hideous
    and time-consuming.
  \item<2-> Even if one is reading and learning hard, from different
    perspective (books and papers), lack of intense discussion can
    hardly create any environment for creative thinking, which is
    desperately needed in basic sciences.
  \item<3-> The advanced topics can be a slippery slope. When one do
    get some original ideas (questions), with no one but the internet
    to turn to, it's really hard to verify or overthrow the idea.
  \item<4-> ...
    
  \end{itemize}
  
\end{frame}
%---------------------------------------------------------




%=========================================================
\section{Attractive solution: YAT-SEN School}

%---------------------------------------------------------
\begin{frame}
  \frametitle{What I see in YAT-SEN School}

  \begin{itemize}
%  \item<1-> What we tell other people:
%    \begin{alertblock}{}
%      We are trying to understand the basic laws of nature, and hence
%      possibly make a difference somewhere.
%    \end{alertblock}
%  \item<2-> The truth:
%    \begin{alertblock}{}
%      It's for fun. And the nature is sophisticated for that fun to
%      last.  
%    \end{alertblock}
  \item<1-> More flexible elective courses
  \item<2-> Tutorial teaching system
  \item<3-> Chances to face rigorous academic challenges on a steady
    basis
  \item<4-> Interactive classes facilitate one's learning in a way that
    just isn't possible in a lecture 
  \end{itemize}
\end{frame}
%---------------------------------------------------------




%=========================================================
%Thanks
\section{}
\begin{frame}
  \begin{center}
    \Huge{\alert{Thanks for your attention!}}\par
    \Huge{\textit{Any questions?}}
  \end{center}
  
\end{frame}

\end{document}

